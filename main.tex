\documentclass[20pt, a1paper, portrait]{tikzposter}
\usetheme{Default}
\tikzposterlatexaffectionproofoff%
\definecolor{tublue}{RGB}{0,102,153}
\colorlet{blocktitlebgcolor}{tublue}
\definecolor{tulightblue}{RGB}{166,213,236}
\colorlet{backgroundcolor}{tulightblue}

\newcommand{\mytitle}{DMFT on the real frequency axis using Exact Diagonalization and Restricted Active Space}
\title{\parbox{0.83\linewidth}{\mytitle}}
\author{Frank T. Ebel\textsuperscript{*}, Karsten Held \color{red}{add Martin, Markus?}}
\institute{
    Institute of Solid State Physics, TU Wien, 1040 Vienna, Austria
    \hspace{5cm}
    \textsuperscript{*}\url{frank.ebel@student.tuwien.ac.at}
}

% font, typesetting
\usepackage{fontspec}
\usepackage[main=english,ngerman]{babel}
\usepackage{microtype}
\usepackage{pdfrender}
\usepackage{pifont} %cross

% math and science typesetting
\usepackage{amsmath}
\usepackage{amssymb}
\usepackage{mathtools}
\usepackage[warnings-off={mathtools-colon,mathtools-overbracket}]{unicode-math} % load after ams
\unimathsetup{math-style=ISO}

% graphics, colors, captions, tables
\usepackage{caption}
\usepackage{float}
\usepackage{graphicx}
\usepackage{pgfplots}
\usepgfplotslibrary{colorbrewer}
\usepgfplotslibrary{groupplots}
\pgfplotsset{
    compat=1.18,
    cycle list/Set1,
    every axis/.append style=
    {
        line width=2pt,
    },
    every axis plot/.append style=
    {
        line width=3.5pt,
    },
    every tick/.append style=
    {
        color=black,
        line width=2pt,
    },
}
\usepackage{pgfplotstable}
\pgfplotstableset{col sep=comma}
\usepackage[labelformat=simple]{subcaption}
\usepackage{tabularx}
\usepackage{tikz}
\usetikzlibrary{arrows.meta}
\usetikzlibrary{fit}
\usetikzlibrary{positioning}
\usetikzlibrary{shapes.symbols}
\tikzset{font={\fontsize{20pt}{12}\selectfont}}
\usepackage{xcolor}
\renewcommand\thesubfigure{(\alph{subfigure})}

% references, bibliography, and links
\usepackage%
[
    backend=biber,
    style=phys,
    biblabel=brackets,
]
{biblatex}
\usepackage{csquotes} % required by biblatex
\usepackage{hyperref}
\addbibresource{ref.bib}
\AtEveryBibitem{\clearfield{title}} % don't print title
\hypersetup{
    % colorlinks=true,
    % urlcolor=blue,
    % citecolor=blue,
    % linkcolor=blue,
    breaklinks=true,
    pdftitle={\mytitle},
    pdfauthor={Frank T.\ Ebel},
}

% misc
\usepackage{shellesc}
\usepackage{multicol}

% commands
\newcommand{\fe}[1]{{\color{red} (#1)}}
\DeclarePairedDelimiter\ket{\lvert}{\rangle}%
\AtBeginDocument{\renewcommand\Re{\mathrm{Re}\>}}
\AtBeginDocument{\renewcommand\Im{\mathrm{Im}\>}}
\newcommand*{\boldcheckmark}{%
  \textpdfrender{
    TextRenderingMode=FillStroke,
    LineWidth=.5pt, % half of the line width is outside the normal glyph
  }{\checkmark}%
}

\begin{document}

\maketitle[width=0.8\textwidth]

\begin{columns}
    % first column
    \column{0.4}
    \block{DMFT impurity solver}
    {
        \begin{itemize}
            \item solve Anderson impurity model for DMFT
            \item real frequencies to avoid analytic continuation
            \item exact diagonalization cannot handle many sites\\
                  $\rightarrow$ idea:
                  add more sites but do not consider all possible states
            \item Lanczos-based method coupling impurity to hundreds of bath sites~\cite{Lu2014,Lu2019}
            \item represent functions using poles without any broadening
                  \begin{equation*}
                      f(\omega) = \sum_i \frac{w_i}{\omega +\mathrm{i}0^+-\epsilon_i}
                  \end{equation*}
            \item Dyson equation gives nonphysical self-energy for discrete bath\\
                  $\rightarrow$ use improved symmetric self-energy estimator $\Sigma^\mathrm{IFG}$~\cite{Kugler2022}
                  using Schur complement \fe{TODO\@: cite Aleksandrs}
            \item goal: look at metal-insulator transition
                  on the Bethe lattice
                  by varying Coulomb interaction $U$ and compare against NRG
                  at $T=0$
        \end{itemize}
    }

    \block{Spectral function $A(\omega)$}
    {
        \begin{itemize}
            \item non-interacting density $A_0(\omega) \propto \sqrt{D^2-\omega^2}$
            \item[\color{green}$\boldcheckmark$] sharp peaks at inner edge of Hubbard bands
            \item[\color{red}\ding{55}] pinning for metal $\muppi DA(\omega=0) = 2$
        \end{itemize}
        \centering
        %! TeX root = ../../main.tex

\begin{tikzpicture}
    \begin{axis}[
            width=9.5cm,
            xmin=-4,
            xmax=4,
            ymin=0,
            ymax=2,
            xlabel=$\omega/D$,
            ylabel=$\muppi D A(\omega)$,
            xtick distance=2.0,
            ytick distance=0.5,
            legend entries={
                    $U=2.4D$,
                    $U=3.5D$,
                },
        ]

        \addplot+ table [x=omega, y=U2.4] {data/spectrum.csv};
        \addplot+ table [x=omega, y=U3.5] {data/spectrum.csv};
    \end{axis}
\end{tikzpicture}

    }

    \block{References}
    {
        \printbibliography[heading=none]
        \textbf{Acknowledgements}\\
        We are grateful for discussions with Maurits W. Haverkort and Aleksandrs Začinskis.
        We thank Fabian B. Kugler for NRG data.
    }

    % second column
    \column{0.6}

    \block{Exact diagonalization with restricted active space}
    {
        \begin{multicols}{2}
            %! TeX root = ../../main.tex

\begin{tikzpicture}[
        inner sep=0.5em,
        node distance=0.5em,
        impurity/.style={
                rectangle,
                inner sep=0pt,
                minimum size=1.5em,
                draw=black,
                path picture={
                        \fill[color=Set1-B] (path picture bounding box.south east)
                        rectangle ($(path picture bounding box.south west)!#1!
                            (path picture bounding box.north west)$);
                    }
            },
        bath/.style={
                circle,
                inner sep=0pt,
                minimum size=1.5em,
                draw=black,
                path picture={
                        \fill[color=Set1-B] (path picture bounding box.south east)
                        rectangle ($(path picture bounding box.south west)!#1!
                            (path picture bounding box.north west)$);
                    }
            },
        line width=0.05em,
    ]

    % sites
    \node [impurity={0.5}] (impurity) {};
    \node [bath={0.5}] (mirror) [right=of impurity] {};
    \foreach \i [remember=\i as \lasti] in {1,...,6}{
            \ifnum\i=1
                \node [bath={0.85}] (v\i) [below=of mirror] {};
                \node [bath={0.15}] (c\i) [above=of mirror] {};
            \else
                \ifnum\i=2
                    \node [bath={0.9}] (v\i) [right=of v\lasti] {};
                    \node [bath={0.1}] (c\i) [right=of c\lasti] {};
                \else
                    \ifnum\i=4
                        \node [bath={1.0}] (v\i) [right=1.5em of v\lasti] {};
                        \node [bath={0.0}] (c\i) [right=1.5em of c\lasti] {};
                    \else
                        \node [bath={1.0}] (v\i) [right=of v\lasti] {};
                        \node [bath={0.0}] (c\i) [right=of c\lasti] {};
                    \fi
                \fi
            \fi
        }

    % labels (separate node for box fit below)
    \node [inner sep=0pt](impuritylabel)[left=0.2em of impurity] {$i$};
    \node [inner sep=0pt](mirrorlabel)[right=0.2em of mirror] {$m$};
    \node (v1label)[below=0mm of v1,inner sep=1mm] {$v_1$};
    \node (v3label)[below=0mm of v3,inner sep=1mm] {$v_{L\vphantom{+1}}$}; % phantom for bounding box size
    \node (v4label)[below=0mm of v4,inner sep=1mm] {$v_{L+1}$};
    \node (c1label)[above=0mm of c1,inner sep=1mm] {$c_1$};
    \node (c3label)[above=0mm of c3,inner sep=1mm] {$c_{L\vphantom{+1}}$}; % phantom for bounding box size
    \node (c4label)[above=0mm of c4,inner sep=1mm] {$c_{L+1}$};


    % connect sites
    \draw (impurity.east) to (mirror.west);
    \draw (impurity.south) to [out=270,in=180] (v1.west);
    \draw (impurity.north) to [out=90,in=180] (c1.west);
    \draw (mirror.south) to (v1.north);
    \draw (mirror.north) to (c1.south);
    \foreach \i [remember=\i as \lasti (initially 1)] in {2,...,6}{
            \draw (v\lasti.east) -- (v\i.west);
            \draw (c\lasti.east) -- (c\i.west);
        }

    % bounding boxes
    \node [
        draw,
        rectangle,
        rounded corners,
        fit=(impuritylabel) (v3label) (c3label) (v3) (c3),
        label=below:$\ket{\phi_I}$,
        line width=0.05em,
        inner sep=0.3em,
    ] {};
    \node [
        draw,
        rectangle,
        rounded corners,
        fit=(v4label) (c4label) (v4) (c4) (v6) (c6),
        label=below:$\ket{\phi_{II}}$,
        line width=0.05em,
        inner sep=0.3em,
    ] {};
\end{tikzpicture}

            \begin{itemize}
                \item natural impurity orbitals~\cite{Lu2014,Lu2019}
                \item split system into two components
                \item $\ket{\phi_I}$: few sites solved by exact diagonalization
                \item $\ket{\phi_{II}}$: many sites restricted to at most $p$ excitations,
                      e.g.\ $p=1$: one hole in $v_j$ xor one particle in $c_j$
            \end{itemize}
        \end{multicols}
    }

    \block{Self-energy $\Sigma(\omega)$}
    {
        \begin{center}
            \hspace{2cm}
            \textcolor{green}{$\boldcheckmark$} divergence for insulator at $\omega=0$
            \hspace{2cm}
            \textcolor{green}{$\boldcheckmark$} $\Im\Sigma(\omega=0)=0$ for metal
            \vspace{0.5cm}
            %! TeX root = ../../main.tex

\begin{tikzpicture}
    \begin{groupplot}[
            group style={
                    group size=2 by 1,
                    horizontal sep=18mm,
                },
            width=7.5cm,
        ]

        \nextgroupplot [
            xmin=-4,
            xmax=4,
            ymin=-10,
            ymax=10,
            xlabel=$\omega/D$,
            ylabel=$(\Re\Sigma - \Sigma^\mathrm{H})/D$,
            xtick distance=2.0,
            ytick distance=5.0,
            legend entries=
                {
                    $U=2.4D$,
                    $U=3.5D$,
                },
            legend style=
                {
                    legend pos=south east,
                },
        ]
        \addplot+ table [x=omega, y=ReU2.4] {data/self-energy.csv};
        \addplot+ [
            % show divergence
            unbounded coords=jump,
            x filter/.expression={abs(x) < 0.02 ? nan : x},
        ]
        table [x=omega, y=ReU3.5] {data/self-energy.csv};
        \node [right] at (rel axis cs:0.02,0.9) {$\delta_\mathrm{Gauss}=0.04D$};

        \nextgroupplot [
            xmin=-4,
            xmax=4,
            ymin=0,
            ymax=10,
            xlabel=$\omega/D$,
            ylabel=$-\Im\Sigma/D$,
            xtick distance=2.0,
            ytick distance=2.5,
        ]
        \addplot+ table [x=omega, y=-ImU2.4] {data/self-energy.csv};
        \addplot+ table [x=omega, y=-ImU3.5] {data/self-energy.csv};

    \end{groupplot}
\end{tikzpicture}
%
        \end{center}
    }

    \block{Metal-insulator transition}
    {
        \begin{itemize}
            \item compare against NRG~\cite{Gauvin-Ndiaye2025}
                  \hspace{8cm}
                  $Z = \frac{m_\mathrm{e}}{m^*} = \left(1 - \frac{\partial\Re\Sigma(0)}{\partial\omega}\right)^{-1}$
        \end{itemize}
        \vspace{0.5cm}
        \centering
        %! TeX root = ../../main.tex

\begin{tikzpicture}
    \begin{groupplot}[
            group style={
                    group size=2 by 1,
                    horizontal sep=18mm,
                },
            width=7.5cm,
            cycle multiindex* list={
                    Set1\nextlist%
                    mark list*\nextlist
                },
            xmin=2.35,
            xmax=3.45,
            xtick distance=0.2,
            yticklabel style={
                    /pgf/number format/fixed,
                },
        ]

        \nextgroupplot [
            ymin=0,
            title=double occupation,
            xlabel=$U/D$,
            ylabel=$\langle n_\uparrow n_\downarrow\rangle$,
            ytick distance=0.02,
            scaled y ticks=false,
            unbounded coords=discard,
        ]
        \addplot [color=gray, mark=*] table [x=U, y=D_NRG_met] {data/D.csv};
        \addplot [color=gray, mark=*, forget plot] table [x=U, y=D_NRG_ins] {data/D.csv};
        \addplot [color=gray, mark=none, dashed, forget plot] coordinates {
                (2.8,0.0255)
                (2.95,0.016)
            };
        \addplot [color=gray, mark=none, dashed, forget plot] coordinates {
                (2.37,0.056)
                (2.37,0.0275)
            };
        \pgfplotsset{cycle list shift=-1}
        % \addplot+ table [x=U, y=D_L1_p1_lin] {data/D.csv};
        % \addplot+ table [x=U, y=D_L1_p1_log] {data/D.csv};
        % \addplot+ table [x=U, y=D_L2_p1_lin] {data/D.csv};
        % \addplot+ table [x=U, y=D_L2_p1_log] {data/D.csv};
        \addplot+ table [x=U, y=D_L1_p2_lin] {data/D.csv};
        \addplot+ table [x=U, y=D_L1_p2_log] {data/D.csv};
        \addplot+ table [x=U, y=D_L2_p2_lin] {data/D.csv};

        \nextgroupplot [
            ymin=0,
            title=quasiparticle weight,
            xlabel=$U/D$,
            ylabel=$Z$,
            ytick distance=0.04,
            legend entries=
                {
                    NRG,
                    {$L=1$, $p=2$, lin.},
                    {$L=1$, $p=2$, log.},
                    {$L=2$, $p=2$, lin.},
                },
            legend cell align=left,
        ]
        \addplot [color=gray, mark=*] table [x=U, y=Z_NRG] {data/Z.csv};
        \addplot [color=gray, mark=*, dashed, forget plot] coordinates {
                (2.8,0.0252)
                (2.95,0.0)
            };
        \pgfplotsset{cycle list shift=-1}
        % \addplot+ table [x=U, y=Z_L1_p1_lin] {data/Z.csv};
        % \addplot+ table [x=U, y=Z_L1_p1_log] {data/Z.csv};
        % \addplot+ table [x=U, y=Z_L2_p1_lin] {data/Z.csv};
        % \addplot+ table [x=U, y=Z_L2_p1_log] {data/Z.csv};
        \addplot+ table [x=U, y=Z_L1_p2_lin] {data/Z.csv};
        \addplot+ table [x=U, y=Z_L1_p2_log] {data/Z.csv};
        \addplot+ table [x=U, y=Z_L2_p2_lin] {data/Z.csv};

    \end{groupplot}
\end{tikzpicture}

    }

    \block{Summary}
    {
        \begin{itemize}
            \item works quite well to calculate
                  spectrum,
                  self-energy,
                  double occupation,
                  quasiparticle weight
            \item exaggerates metallic region somewhat $U_{c2}\approx3.2D$
            \item outlook: investigate the coexistence region and obtain $U_{c1}$
        \end{itemize}
    }

\end{columns}

\end{document}
