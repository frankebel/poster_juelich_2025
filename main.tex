\documentclass{article}

\usepackage[a0paper,portrait]{geometry}
\usepackage[poster]{tcolorbox}
\pagestyle{empty}
\def\lcol{0.8} % width of left column
\def\rcol{1.2} % width of right column


\title{DMFT on the real frequency axis using exact diagonalization and restricted active space}
\author{Frank T. Ebel, Martin Braß, Markus Wallerberger, and Karsten Held}

% font, typesetting
\usepackage[fontsize=34]{fontsize}
\usepackage{fontspec}
\usepackage{enumitem}
\setitemize{itemsep=1pt}
\usepackage[main=english]{babel}
\usepackage{microtype}
\usepackage{pdfrender}
\usepackage{pifont} %cross
\renewcommand{\familydefault}{\sfdefault} % sans-serif

% math and science typesetting
\usepackage{amsmath}
\usepackage{amssymb}
\usepackage{mathtools}
\usepackage[warnings-off={mathtools-colon,mathtools-overbracket}]{unicode-math} % load after ams
\unimathsetup{math-style=ISO}

% graphics, colors, captions, tables
\usepackage{caption}
\usepackage{float}
\usepackage{graphicx}
\usepackage{pgfplots}
\usepgfplotslibrary{colorbrewer}
\usepgfplotslibrary{groupplots}
\pgfplotsset{
    compat=1.18,
    cycle list/Set1,
    every axis/.append style=
    {
        line width=4pt,
        legend style={
            draw=none,
        },
        ylabel shift=-0.5em,
    },
    every axis plot/.append style=
    {
        line width=5pt,
    },
    every tick/.append style=
    {
        color=black,
        line width=3pt,
    },
}
\usepackage{pgfplotstable}
\pgfplotstableset{col sep=comma}
\usepackage[labelformat=simple]{subcaption}
\usepackage{tabularx}
\usepackage{tikz}
\usetikzlibrary{
    arrows.meta,
    fit,
    positioning,
    shapes.symbols,
}
\tikzset{font={\fontsize{30pt}{2}\selectfont}}
\usepackage{xcolor}
\definecolor{tublue}{RGB}{0,102,153}
\definecolor{tulightblue}{RGB}{166,213,236}

% references, bibliography, and links
\usepackage[
    backend=biber,
    style=phys,
    biblabel=brackets,
    maxnames=2,
]{biblatex}
\usepackage{csquotes} % required by biblatex
\usepackage{hyperref}
\addbibresource{ref.bib}
\AtEveryBibitem{\clearfield{title}} % don't print title
\hypersetup{
    % colorlinks=true,
    % urlcolor=blue,
    % citecolor=blue,
    % linkcolor=blue,
    breaklinks=true,
    pdftitle={\@title},
    pdfauthor={Frank T.\ Ebel},
}

% commands
\DeclarePairedDelimiter\ket{\lvert}{\rangle}%
\AtBeginDocument{\renewcommand\Re{\mathrm{Re}\>}}
\AtBeginDocument{\renewcommand\Im{\mathrm{Im}\>}}
\newcommand*{\boldcheckmark}{
  \textpdfrender{
    TextRenderingMode=FillStroke,
    LineWidth=.5pt, % half of the line width is outside the normal glyph
  }{\checkmark}%
}
\newcommand{\pk}{\vphantom{k}}
\newcommand{\pdag}{\vphantom{\dag}}
\newcommand{\ps}{\vphantom{*}}
\newcommand{\cemph}[1]{\textcolor{tublue}{#1}} % color emphasize

\begin{document}

\begin{tcbposter}[
        coverage = {
                spread,
                interior style={color=tulightblue}, % background color
            },
        poster = {
                % showframe,
                rows=5,
                columns=2,
                spacing=1cm,
            },
        boxes = {
                arc=3mm,
                boxsep=20pt,
                boxrule=6mm,
                colback=white,
                colframe=tublue,
                title style={tublue},
                fonttitle=\centering\bfseries\Large
            },
    ]

    \posterbox[
        halign=center,
        valign=center,
        % colupper=white,
        % colback=tublue,
    ]{
        name=title,
        column=1,
        span=2,
        below=top,
    }{
        {\bfseries\Huge\@title}\\
        \vspace{10mm}
        {\bfseries\Large\@author}\\
        \vspace{10mm}
        Institute of Solid State Physics, TU Wien, 1040 Vienna, Austria\hspace{5cm}
        \url{frank.ebel@student.tuwien.ac.at}
    }

    \posterbox[adjusted title=Problem]{
        name=problem,
        column=1,
        span=\lcol,
        below=title,
    }{
        \begin{itemize}[leftmargin=*]
            \item solve \cemph{Anderson impurity model} for DMFT
                  \begin{align*}
                      H
                       & =
                      U n_\uparrow n_\downarrow
                      +
                      (\epsilon - \mu) \sum_{\pk\sigma} n_\sigma
                      +
                      \sum_{k\sigma} \epsilon_{k\sigma}^{\pdag} c^\dag_{k\sigma} c^{\pdag}_{k\sigma} \\
                       & +
                      \sum_{k\sigma} \left(V^{\ps}_{k\sigma} d^\dag_{\pk\sigma} c^{\pdag}_{k\sigma}
                      + V_{j\sigma}^* c^\dag_{k\sigma} d^{\pdag}_{\pk\sigma} \right)
                  \end{align*}
            \item \cemph{real frequencies} to avoid analytic continuation
            \item exact diagonalization cannot handle many sites $N$,
                  binomial scaling at half-filling
                  \begin{equation*}
                      \binom{2N}{N} \approx \frac{4^N}{\sqrt{\pi N}}
                  \end{equation*}
            \item \cemph{idea:} add more sites but do not consider all possible states,
                  polynomial scaling $N^p$ (restricted active space)
            \item \cemph{goal:} look at metal-insulator transition
                  on the Bethe lattice
                  by varying Coulomb interaction $U$ and compare against NRG
                  at $T=0$
        \end{itemize}
    }

    \posterbox[adjusted title=Method]{
        name=method,
        column*=2,
        span=\rcol,
        below=title,
    }{
        \begin{itemize}[leftmargin=*]
            \item Lanczos algorithm with restricted active space
                  coupling impurity to hundreds of bath sites~\cite{Lu2014,Lu2019}
            \item represent functions (e.g.\ correlators, self-energy)
                  using poles without any broadening
                  \begin{equation*}
                      f(\omega) = f_0 + \sum_j \frac{w_j}{\omega +\mathrm{i}0^+-\epsilon_j}
                  \end{equation*}
        \end{itemize}
        \vspace{-1.4em}
        \begin{minipage}[t]{0.55\linewidth}
            \begin{itemize}[leftmargin=*]
                \item use natural impurity orbitals~\cite{Lu2014,Lu2019}
                \item split system into two components:
                      \vspace{-0.6em}
                      \begin{itemize}[leftmargin=*]
                          \item $\ket{\phi_I}$: few sites solved by exact diagonalization
                          \item $\ket{\phi_{II}}$: many sites restricted to at most $p$ excitations,
                                e.g.\ $p=1$: one hole in $v_j$ xor one particle in $c_j$
                      \end{itemize}
            \end{itemize}
        \end{minipage}%
        \begin{minipage}[t]{0.45\linewidth}
            \raggedleft%
            \vspace*{0pt}
            %! TeX root = ../../main.tex

\begin{tikzpicture}[
        inner sep=0.5em,
        node distance=0.5em,
        impurity/.style={
                rectangle,
                inner sep=0pt,
                minimum size=1.5em,
                draw=black,
                path picture={
                        \fill[color=Set1-B] (path picture bounding box.south east)
                        rectangle ($(path picture bounding box.south west)!#1!
                            (path picture bounding box.north west)$);
                    }
            },
        bath/.style={
                circle,
                inner sep=0pt,
                minimum size=1.5em,
                draw=black,
                path picture={
                        \fill[color=Set1-B] (path picture bounding box.south east)
                        rectangle ($(path picture bounding box.south west)!#1!
                            (path picture bounding box.north west)$);
                    }
            },
        line width=0.05em,
    ]

    % sites
    \node [impurity={0.5}] (impurity) {};
    \node [bath={0.5}] (mirror) [right=of impurity] {};
    \foreach \i [remember=\i as \lasti] in {1,...,6}{
            \ifnum\i=1
                \node [bath={0.85}] (v\i) [below=of mirror] {};
                \node [bath={0.15}] (c\i) [above=of mirror] {};
            \else
                \ifnum\i=2
                    \node [bath={0.9}] (v\i) [right=of v\lasti] {};
                    \node [bath={0.1}] (c\i) [right=of c\lasti] {};
                \else
                    \ifnum\i=4
                        \node [bath={1.0}] (v\i) [right=1.5em of v\lasti] {};
                        \node [bath={0.0}] (c\i) [right=1.5em of c\lasti] {};
                    \else
                        \node [bath={1.0}] (v\i) [right=of v\lasti] {};
                        \node [bath={0.0}] (c\i) [right=of c\lasti] {};
                    \fi
                \fi
            \fi
        }

    % labels (separate node for box fit below)
    \node [inner sep=0pt](impuritylabel)[left=0.2em of impurity] {$i$};
    \node [inner sep=0pt](mirrorlabel)[right=0.2em of mirror] {$m$};
    \node (v1label)[below=0mm of v1,inner sep=1mm] {$v_1$};
    \node (v3label)[below=0mm of v3,inner sep=1mm] {$v_{L\vphantom{+1}}$}; % phantom for bounding box size
    \node (v4label)[below=0mm of v4,inner sep=1mm] {$v_{L+1}$};
    \node (c1label)[above=0mm of c1,inner sep=1mm] {$c_1$};
    \node (c3label)[above=0mm of c3,inner sep=1mm] {$c_{L\vphantom{+1}}$}; % phantom for bounding box size
    \node (c4label)[above=0mm of c4,inner sep=1mm] {$c_{L+1}$};


    % connect sites
    \draw (impurity.east) to (mirror.west);
    \draw (impurity.south) to [out=270,in=180] (v1.west);
    \draw (impurity.north) to [out=90,in=180] (c1.west);
    \draw (mirror.south) to (v1.north);
    \draw (mirror.north) to (c1.south);
    \foreach \i [remember=\i as \lasti (initially 1)] in {2,...,6}{
            \draw (v\lasti.east) -- (v\i.west);
            \draw (c\lasti.east) -- (c\i.west);
        }

    % bounding boxes
    \node [
        draw,
        rectangle,
        rounded corners,
        fit=(impuritylabel) (v3label) (c3label) (v3) (c3),
        label=below:$\ket{\phi_I}$,
        line width=0.05em,
        inner sep=0.3em,
    ] {};
    \node [
        draw,
        rectangle,
        rounded corners,
        fit=(v4label) (c4label) (v4) (c4) (v6) (c6),
        label=below:$\ket{\phi_{II}}$,
        line width=0.05em,
        inner sep=0.3em,
    ] {};
\end{tikzpicture}

        \end{minipage}
    }

    \posterbox[adjusted title=Spectral function $A(\omega)$]{
        name=spectral-function,
        column=1,
        span=\lcol,
        below=problem,
    }{
        \begin{itemize}[leftmargin=*]
            \item non-interacting density $A_0(\omega) \propto \sqrt{D^2-\omega^2}$
            \item[\color{green}$\boldcheckmark$] sharp peaks at inner edge of Hubbard bands (cf.~\cite{Ganahl2015})
            \item[\color{red}\ding{55}] pinning for metal $\muppi DA(\omega=0) = 2$
        \end{itemize}
        \centering
        %! TeX root = ../../main.tex

\begin{tikzpicture}
    \begin{axis}[
            width=9.5cm,
            xmin=-4,
            xmax=4,
            ymin=0,
            ymax=2,
            xlabel=$\omega/D$,
            ylabel=$\muppi D A(\omega)$,
            xtick distance=2.0,
            ytick distance=0.5,
            legend entries={
                    $U=2.4D$,
                    $U=3.5D$,
                },
        ]

        \addplot+ table [x=omega, y=U2.4] {data/spectrum.csv};
        \addplot+ table [x=omega, y=U3.5] {data/spectrum.csv};
    \end{axis}
\end{tikzpicture}

    }

    \posterbox[adjusted title=Self-energy $\Sigma(\omega)$]{
        name=self-energy,
        column*=2,
        span=\rcol,
        below=method,
    }{
        Dyson equation gives nonphysical self-energy for discrete bath\\
        \hspace*{1em} $\rightarrow$ use \cemph{symmetric improved self-energy} estimator $\Sigma^\mathrm{IFG}$~\cite{Kugler2022}
        using Schur complement\\
        \vspace{0.1em}\\
        \hspace*{3.4em}
        \textcolor{green}{$\boldcheckmark$} divergence at $\omega=0$ for insulator
        \hspace{4.8em}
        \textcolor{green}{$\boldcheckmark$} $\Im\Sigma(\omega=0)=0$ for metal
        \vspace{-0.5em}
        \begin{center}
            %! TeX root = ../../main.tex

\begin{tikzpicture}
    \begin{groupplot}[
            group style={
                    group size=2 by 1,
                    horizontal sep=18mm,
                },
            width=7.5cm,
        ]

        \nextgroupplot [
            xmin=-4,
            xmax=4,
            ymin=-10,
            ymax=10,
            xlabel=$\omega/D$,
            ylabel=$(\Re\Sigma - \Sigma^\mathrm{H})/D$,
            xtick distance=2.0,
            ytick distance=5.0,
            legend entries=
                {
                    $U=2.4D$,
                    $U=3.5D$,
                },
            legend style=
                {
                    legend pos=south east,
                },
        ]
        \addplot+ table [x=omega, y=ReU2.4] {data/self-energy.csv};
        \addplot+ [
            % show divergence
            unbounded coords=jump,
            x filter/.expression={abs(x) < 0.02 ? nan : x},
        ]
        table [x=omega, y=ReU3.5] {data/self-energy.csv};
        \node [right] at (rel axis cs:0.02,0.9) {$\delta_\mathrm{Gauss}=0.04D$};

        \nextgroupplot [
            xmin=-4,
            xmax=4,
            ymin=0,
            ymax=10,
            xlabel=$\omega/D$,
            ylabel=$-\Im\Sigma/D$,
            xtick distance=2.0,
            ytick distance=2.5,
        ]
        \addplot+ table [x=omega, y=-ImU2.4] {data/self-energy.csv};
        \addplot+ table [x=omega, y=-ImU3.5] {data/self-energy.csv};

    \end{groupplot}
\end{tikzpicture}

        \end{center}
    }

    \posterbox[adjusted title=Metal-insulator transition]{
        name=metal-insulator-transition,
        column*=2,
        span=\rcol,
        below=self-energy,
    }{
        \begin{itemize}[leftmargin=*]
            \item compare against NRG~\cite{Gauvin-Ndiaye2025}
                  \hspace{0.35\textwidth}
                  $Z = \frac{m_\mathrm{e}}{m^*} = \left(1 - \frac{\partial\Re\Sigma(0)}{\partial\omega}\right)^{-1}$
        \end{itemize}
        \vspace{-0.5em}
        \centering
        %! TeX root = ../../main.tex

\begin{tikzpicture}
    \begin{groupplot}[
            group style={
                    group size=2 by 1,
                    horizontal sep=18mm,
                },
            width=7.5cm,
            cycle multiindex* list={
                    Set1\nextlist%
                    mark list*\nextlist
                },
            xmin=2.35,
            xmax=3.45,
            xtick distance=0.2,
            yticklabel style={
                    /pgf/number format/fixed,
                },
        ]

        \nextgroupplot [
            ymin=0,
            title=double occupation,
            xlabel=$U/D$,
            ylabel=$\langle n_\uparrow n_\downarrow\rangle$,
            ytick distance=0.02,
            scaled y ticks=false,
            unbounded coords=discard,
        ]
        \addplot [color=gray, mark=*] table [x=U, y=D_NRG_met] {data/D.csv};
        \addplot [color=gray, mark=*, forget plot] table [x=U, y=D_NRG_ins] {data/D.csv};
        \addplot [color=gray, mark=none, dashed, forget plot] coordinates {
                (2.8,0.0255)
                (2.95,0.016)
            };
        \addplot [color=gray, mark=none, dashed, forget plot] coordinates {
                (2.37,0.056)
                (2.37,0.0275)
            };
        \pgfplotsset{cycle list shift=-1}
        % \addplot+ table [x=U, y=D_L1_p1_lin] {data/D.csv};
        % \addplot+ table [x=U, y=D_L1_p1_log] {data/D.csv};
        % \addplot+ table [x=U, y=D_L2_p1_lin] {data/D.csv};
        % \addplot+ table [x=U, y=D_L2_p1_log] {data/D.csv};
        \addplot+ table [x=U, y=D_L1_p2_lin] {data/D.csv};
        \addplot+ table [x=U, y=D_L1_p2_log] {data/D.csv};
        \addplot+ table [x=U, y=D_L2_p2_lin] {data/D.csv};

        \nextgroupplot [
            ymin=0,
            title=quasiparticle weight,
            xlabel=$U/D$,
            ylabel=$Z$,
            ytick distance=0.04,
            legend entries=
                {
                    NRG,
                    {$L=1$, $p=2$, lin.},
                    {$L=1$, $p=2$, log.},
                    {$L=2$, $p=2$, lin.},
                },
            legend cell align=left,
        ]
        \addplot [color=gray, mark=*] table [x=U, y=Z_NRG] {data/Z.csv};
        \addplot [color=gray, mark=*, dashed, forget plot] coordinates {
                (2.8,0.0252)
                (2.95,0.0)
            };
        \pgfplotsset{cycle list shift=-1}
        % \addplot+ table [x=U, y=Z_L1_p1_lin] {data/Z.csv};
        % \addplot+ table [x=U, y=Z_L1_p1_log] {data/Z.csv};
        % \addplot+ table [x=U, y=Z_L2_p1_lin] {data/Z.csv};
        % \addplot+ table [x=U, y=Z_L2_p1_log] {data/Z.csv};
        \addplot+ table [x=U, y=Z_L1_p2_lin] {data/Z.csv};
        \addplot+ table [x=U, y=Z_L1_p2_log] {data/Z.csv};
        \addplot+ table [x=U, y=Z_L2_p2_lin] {data/Z.csv};

    \end{groupplot}
\end{tikzpicture}

    }

    \posterbox[adjusted title=Conclusion]{
        name=references,
        column*=2,
        span=\rcol,
        between=metal-insulator-transition and bottom,
    }{
        \begin{itemize}[leftmargin=*]
            \item quantities improve with increased complexity $L\uparrow$, $p\uparrow$
            \item exaggerates metallic region somewhat $U_{c2}\approx3.2D$
            \item outlook: investigate the coexistence region and obtain $U_{c1}$
        \end{itemize}
    }

    \posterbox[adjusted title=References]{
        name=references,
        column=1,
        span=\lcol,
        between=spectral-function and bottom,
    }{
        \printbibliography[heading=none]
        \textbf{Acknowledgements}\\
        We are grateful for fruitful discussions with Maurits W. Haverkort and Aleksandrs Začinskis.
        We thank Fabian B. Kugler for NRG data.
    }

\end{tcbposter}

\end{document}
